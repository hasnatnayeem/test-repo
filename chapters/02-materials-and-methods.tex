\chapter{Materials and Methods}
This chapter will present the materials and methods employed to investigate the secondary metabolites of the fungal strain Cyclocybe aegerita AAE-3, which is classified under the genus Cyclocybe.

\section{Materials}






\definecolor{Black}{rgb}{255,255,255}
\begin{longtblr}[
  caption = {Culture media applied for the cultivation of fungal strain AAE-3},
]{
  width = \linewidth,
  colspec = {Q[202]Q[204]Q[75]Q[262]Q[125]Q[67]},
  cell{2}{1} = {r=8}{},
  cell{2}{5} = {r=8}{},
  cell{2}{6} = {r=8}{},
  cell{10}{1} = {r=3}{},
  cell{10}{5} = {r=3}{},
  cell{10}{6} = {r=3}{},
  cell{13}{1} = {r=3}{},
  cell{13}{5} = {r=3}{},
  cell{13}{6} = {r=3}{},
  cell{17}{1} = {r=9}{},
  cell{17}{5} = {r=9}{},
  cell{17}{6} = {r=9}{},
  cell{26}{1} = {r=3}{},
  cell{26}{5} = {r=3}{},
  cell{26}{6} = {r=3}{},
  cell{29}{1} = {r=4}{},
  cell{29}{5} = {r=4}{},
  cell{29}{6} = {r=4}{},
  cell{33}{1} = {r=4}{},
  cell{33}{5} = {r=4}{},
  cell{33}{6} = {r=4}{},
  cell{37}{1} = {r=4}{},
  cell{37}{5} = {r=4}{},
  cell{37}{6} = {r=4}{},
  vlines = {white},
  vline{1} = {1-2,10,13,16-17,26,29,33,37}{white},
  hline{1-2,10,13,16-17,26,29,33,37,41} = {-}{},
  % hline{3-9,11-12,14-15,18-25,27-28,30-32,34-36,38-40} = {2-4}{},
}
\textbf{Medium} & \textbf{Component} & {\textbf{Weight}\\\textbf{(g/L)}} & \textbf{Manufacturer} & \textbf{Type} & \textbf{pH}\\
ZM ½ & Molasses & 5 & Nordzucker AG & Liquid & 7.2\\
 & Oatmeal & 5 & Herrnmuehle &  & \\
 & Sucrose & 4 & Carl Roth GmbH + Co. KG &  & \\
 & Mannitol & 4 & AppliChem GmbH &  & \\
 & D-Glucose & 1.5 & Cerestar GmbH &  & \\
 & CaCO3 & 1.5 & Carl Roth GmbH + Co. KG &  & \\
 & {Edamin\\(Lactalbumin~\\Hydrolysate)} & 0.5 & Thermo Fisher Scientific Oxoid Ltd &  & \\
 & (NH4)2SO4 & 0.5 & Merck KGaA &  & \\
{Q6 ½\\(Cottonseed-Medium)} & D-Glucose & 2.5 & CerestarGmbH & Liquid & 7.2\\
 & Glycerol & 10 & Carl Roth GmbH + Co. KG &  & \\
 & {Cotton seed~\\flour} & 5 & Sigma-Aldrich Chemie GmbH &  & \\
{YM 6.3\\(Yeast-Malt-Medium)} & Malt extract & 10 & Organotechnie & Liquid~ & 6.3\\
 & D-Glucose & 4 & Cerestar Deutschland GmbH &  & \\
 & Yeast extract & 4 & Ohly GmbH &  & \\
MEA & {Malt extract\\Ager} & 15 & Organotechnie & Liquid~ and solid & 6.2\\
MGP & D-Glucose & 10 & Cerestar Deutschland GmbH & Liquid~ & ~ ~ ~ ~ ~ -\\
 & Maltos & 20 & ThermoFisher GmbH &  & \\
 & Soya peptone & 2 & Sigma-Aldrich Chemie GmbH &  & \\
 & Yeast extract & 1 & Herrnmühle; Reichelsheim,Germany &  & \\
 & KH2PO4 & 1 & Merck KGaA &  & \\
 & MgSO4 * 7H2O & 0.5 & Merck KGaA &  & \\
 & Fecl3(10mM) & 1 ml & Merck KGaA &  & \\
 & ZnSO4(1.78g/l) & 1 ml & MP Biomedicals Germany GmbH &  & \\
 & Cacl2(0.1M) & 1 ml & Merck KGaA &  & \\
SYM & Malt extract & 30 & Organotechnie & Liquid & 6.3\\
 & Sucrose & 10 & {Carl Roth GmbH  Co. KG;\\Karlsruhe, Germany} &  & \\
 & Yeast extract & 4 & Herrnmühle; Reichelsheim, Germany &  & \\
{MR\\(Modified Robbins medium)} & D-Glucose & 40 & Cerestar Deutschland GmbH & Liquid~ & 5.8\\
 & Czapek Dox-broth & 35 & Sigma-Aldrich Chemie GmbH &  & \\
 & Corn steep liquor & 5 & Sigma-Aldrich Chemie GmbH &  & \\
 & Poppers tree wood extract & 30 &  &  & \\
BRFT & Brown rice & 280 & Kaufland Bio Langkorn Naturreis & Solid & ~ ~ -\\
 & {Liquid part:\\Yeast extract} & 1 & Ohly GmbH &  & \\
 & di-Sodium tartrate dihydrate & 0.5 & Merck KGaA &  & \\
 & KH2PO4 & 0.5 & Carl Roth GmbH + Co. KG &  & \\ 
OAT & Oat grains & 350 & Kaufland Bio Hafer Ganzes korn & Solid & ~ ~ ~ ~ -\\
 & {Liquid part:\\Yeast extract} & 1 & Ohly GmbH &  & \\
 & di-Sodium tartrate dihydrate & 0.5 & Merck KGaA &  & \\
 & KH2PO4 & 0.5 & Carl Roth GmbH + Co. KG &  & 
\end{longtblr}




% \usepackage{array}
% \usepackage[longtable]{multirow}
% \usepackage{longtable}
% \usepackage{colortbl}


\arrayrulecolor{black}
\begin{longtable}{!{\color{black}}>{\hspace{0pt}}m{0.162\linewidth}!{\color{black}}>{\hspace{0pt}}m{0.188\linewidth}!{\color{black}}>{\hspace{0pt}}m{0.102\linewidth}!{\color{black}}>{\hspace{0pt}}m{0.321\linewidth}!{\color{black}}>{\hspace{0pt}}m{0.072\linewidth}!{\color{black}}>{\hspace{0pt}}m{0.094\linewidth}!{\color{black}}}
\caption{Media used for bioactivity testing}\\ 
\hline
\textbf{Medium} & \textbf{Component} & \textbf{Weight~}\par{}\textbf{(g/L)/}\par{}\textbf{volume} & \textbf{Manufacturer} & \textbf{Type} & \textbf{pH} \endfirsthead 
\hline
DMEM & ~ ~ ~ ~ - & ~ ~ ~ ~ - & Gibco,~\par{}Thermo Fisher Scientific Inc.\par{}MA, USA & Liquid & 7 \\* 
\hline
\multirow{2}{0.162\linewidth}{\hspace{0pt}\begin{tabular}[c]{@{}l@{}}Middlebrook\end{tabular}} & Middlebrook 7H9 & 2.35 & Honeywell Fluka;~\par{}Seelze, Germany & \multirow{2}{0.067\linewidth}{\hspace{0pt}Liquid} & \multirow{2}{0.094\linewidth}{\hspace{0pt}6.6±0.2} \\* 
% \cline{2-4}
 & Broth Base\par{}ADC Enrichment\par{}Medium & 10\%~\par{}(v/v) & BD BBL, Becton, Dickinson~\par{}and Company Sparks;\par{}MD, USA &  &  \\ 
\hline
Müller-Hinton-\par{}Bouillon & ~ - & 21 & Carl Roth GmbH \& Co. KG;~\par{}Karlsruhe, Germany & Liquid & 7.4±0.2 \\* 
\hline
\multirow{3}{0.162\linewidth}{\hspace{0pt}\begin{tabular}[c]{@{}l@{}}MYC\end{tabular}} & Phytone peptone & 10 & BD Biosciences;~\par{}NJ, USA & \multirow{3}{0.067\linewidth}{\hspace{0pt}Liquid} & \multirow{3}{0.094\linewidth}{\hspace{0pt}7} \\* 
% \cline{2-4}
 & Glucose & 10 & Cerestar Deutschland Gmbh;~\par{}Krefeld, Germany &  &  \\* 
% \cline{2-4}
 & HEPES & 11.9 & Carl Roth GmbH \& Co. KG;~\par{}Karlsruhe, Germany &  &  \\ 
\hline
F12K & ~ ~ ~ ~ - & ~ ~ ~ ~ - & Gibco,~\par{}Thermo Fisher Scientific Inc.\par{}MA, USA & Liquid & 7 \\ 
\hline
McCoy’s~\par{}5A~\par{}(modified) & ~ ~ ~ - & ~ ~ ~ ~ - & Gibco,~\par{}Thermo Fisher Scientific Inc.\par{}MA, USA & Liquid & 7 \\
\hline
\end{longtable}
\arrayrulecolor{black}









\arrayrulecolor{black}
\begin{longtable}{!{\color{black}}>{\hspace{0pt}}m{0.338\linewidth}!{\color{black}}>{\hspace{0pt}}m{0.602\linewidth}!{\color{black}}}
\caption{Chemicals}\\ 
\hline
\textbf{Chemical} & \textbf{Manufacturer} \endfirsthead 
\hline
Acetone & J.T. Baker, Avantor Performance Materials; PA, USA \\ 
\hline
Acetonitrile~ & J.T. Baker, Avantor Performance Materials; PA, USA \\ 
\hline
Dichloromethane & J.T. Baker, Avantor Performance Materials; PA, USA \\ 
\hline
Ethyl acetate & J.T. Baker, Avantor Performance Materials; PA, USA \\ 
\hline
Formic acid & Honeywell Fluka; Seelze, Germany \\ 
\hline
Heptane & Carl Roth GmbH  Co. KG; Karlsruhe, Germany \\ 
\hline
Methanol & J.T. Baker, Avantor Performance Materials; PA, USA \\ 
\hline
Ultra-pure water & Milli-Q, Lab Water, Merck Chemicals GmbH; Darmstadt \\ 
\hline
TBME (tert-Butyl methyl ether) & Carl Roth GmbH  Co. KG; Karlsruhe, Germany \\ 
\hline
Chloroform & J.T. Baker, Avantor Performance Materials; PA, USA \\ 
\hline
Tetrahydrofuran (THF) & Carl Roth GmbH  Co. KG; Karlsruhe, Germany \\ 
\hline
Isopropanol & J.T. Baker, Avantor Performance Materials; PA, USA \\ 
\hline
Petrolether & Carl Roth GmbH  Co. KG; Karlsruhe, Germany \\ 
\hline
Chloroform-\textit{d6} & Carl Roth GmbH  Co. KG; Karlsruhe, Germany \\
\hline
\end{longtable}
\arrayrulecolor{black}





\arrayrulecolor{black}
\begin{longtable}{!{\color{black}}>{\hspace{0pt}}m{0.220\linewidth}!{\color{black}}>{\hspace{0pt}}m{0.292\linewidth}!{\color{black}}>{\hspace{0pt}}m{0.477\linewidth}!{\color{black}}}
\caption{General Instruments}\\ 
\hline
\textbf{Device} & \textbf{Model} & \textbf{Manufacturer} \endfirsthead 
\hline
Autoclave & VX-150 & Systec GmbH \& Co. KG;\par{}Asbach, Germany \\ 
\hline
Centrifuge & Rotofix 32A & Hettich; Kirchlengern, Germany \\ 
\hline
Diabetes test\newline strips & Medi-Test for glucose & Macherey-Nagel; Düren, Germany \\ 
\hline
Homogenizer & T25 easy clean digital\newline ULTRA -TURRAX® & IKA-Werke;\par{}Staufen im Breisgau, Germany \\ 
\hline
Heating chamber & REACTI-THERM\newline TS-18821 & Thermo Fisher Scientific Inc. MA, USA \\ 
\hline
Shaker incubator & AG15 & Infors HT; Bottmingen, Switzerland \\ 
\hline
Magnetic stirrer & RCT digital & IKA-Werke; Staufen im Breisgau,\par{}Germany \\ 
\hline
Micropipette & 0.5 - 10 µl\par{}10 - 100 µl\par{}100 - 1000 µl & Eppendorf Vertrieb Deutschland GmbH;\par{}Wesseling-Berzdorf, Germany \\ 
\hline
Nitrogen Dryer & VLM & VLM Korrosions-Prüftechnik,\par{}Labortechnik \& Dienstleistungen\par{}GmbH \\ 
\hline
NMR spectrometer & Bruker Avance III 700\par{}(1H 700 MHz, 13C 175 MHz)\par{}Bruker Avance III 500\par{}(1H 500 MHz, 13C 125 MHz) & Bruker Daltonik GmbH;\par{}Bremen, Germany \\ 
\hline
pH meter & inoLab® pH730 & WTW Wissenschaftlich-\par{}Technische Werkstätten GmbH \\ 
\hline
Rotary evaporator\par\null\par\null\par{}Pumping unit & Hei-VAP precision\par\null\par\null\par{}PC 3001 VARIO select & Heidolph Instruments GmbH \& Co. KG;\par{}Schwabach, Germany\par{}Vacuubrand GmbH \& Co. KG;\par{}Wertheim, Germany \\ 
\hline
Ultrasonic bath & Sonorex Digital 10P & Bandelin electronic GmbH \& Co. KG;\par{}Berlin, Germany \\ 
\hline
Ultrapure water\newline system & Milli-Q Direct & Merck Chemicals GmbH; Darmstadt, Germany \\ 
\hline
Parafilm & Parafilm M All-Purpose Laboratory Film & Bemis Company, Inc. Neenah, USA \\ 
\hline
SpeedVac\par{}Vacuum pump & RVC 2-25 CD plus CT 02-50 SR & CHRIST®; Osterode am Harz, Germany \\ 
\hline
Vortex & Vortex 2 & Werke GmbH  Co. KG; Staufen, Germany \\ 
\hline
Clean Bench & MAXISAFE 2020 & Thermo Fisher Scientific\par{}168 Third Avenue\par{}Waltham, MA USA 02451 \\ 
\hline
Pipette tips & Ep T.I.P.S. LoRetentionReloads & Eppendorf Vertrieb Deutschland GmbH; Wesseling-Berzdorf, Germany \\
\hline
\end{longtable}
\arrayrulecolor{black}






\arrayrulecolor{black}
\begin{longtable}{!{\color{black}}>{\hspace{0pt}}m{0.3\linewidth}!{\color{black}}>{\hspace{0pt}}m{0.2\linewidth}!{\color{black}}>{\hspace{0pt}}m{0.44\linewidth}!{\color{black}}}
\caption{Devices used for chromatography}\\ 
\hline
\textbf{Device} & \textbf{Model} & \textbf{Company} \endfirsthead 
\hline
Analytical HPLC system & Agilent 1200 Infinity\par{}Series & Agilent Technologies Deutschland\par{}GmbH; Böblingen, Germany \\ 
\hline
ESI - IT mass spectrometer & AmaZon speedTM & Bruker Daltonics GmbH  Co. KG; Bremen, Germany \\ 
\hline
Flash chromatography system & Reveleris® X2 & W. R. Grace  Co. Columbia; MD, USA \\ 
\hline
HR-ESI mass spectrometer & MaXis & Bruker Daltonics GmbH  Co. KG; Bremen, Germany \\ 
\hline
Trapped Ion Mobility\newline Spectrometry (TIMS) & timsTOF Pro2 & Bruker Daltonics GmbH  Co. KG; Bremen, Germany \\ 
\hline
Preparative HPLC system & Gilson PLC 2250 and 2050 & Gilson, Inc. Middleton; WI, USA \\ 
\hline
Semi Preparative HPLC system & Agilent 1200 Infinity Series & Agilent Technologies Deutschland GmbH; Böblingen, Germany \\
\hline
\end{longtable}
\arrayrulecolor{black}




% \usepackage{array}
% \usepackage[longtable]{multirow}
% \usepackage{longtable}
% \usepackage{colortbl}


\arrayrulecolor{black}
\begin{longtable}{!{\color{black}}>{\hspace{0pt}}m{0.102\linewidth}!{\color{black}}>{\hspace{0pt}}m{0.165\linewidth}!{\color{black}}>{\hspace{0pt}}m{0.098\linewidth}!{\color{black}}>{\hspace{0pt}}m{0.256\linewidth}!{\color{black}}>{\hspace{0pt}}m{0.315\linewidth}!{\color{black}}}
\caption{Fungal strains investigated in this study}\\ 
\hline
\textbf{Genus} & \textbf{Species~} & \textbf{Strain No.} & \textbf{Isolation source} & \textbf{Types} \\* 
\hline
\multirow{5}{0.102\linewidth}{\hspace{0pt}Cyclocybe} & \multirow{5}{\linewidth}{\hspace{0pt}\newline Cyclocybe aegerita} & AAE-3~ & Sent by university of Giessen & Dikaryotic and fruiter type \\* 
% \cline{3-5}
 &  & AAE-3-32 & Sent by university of Giessen & Monokaryotic and fruiter type \\* 
% \cline{3-5}
 &  & AAE-3-24 & Sent by university of Giessen & Monokaryotic and mycelium type \\* 
% \cline{3-5}
 &  & AAE-3-40 & Sent by university of Giessen & Monokaryotic and initials type \\* 
% \cline{3-5}
 &  & AAE-3-37 & Sent by university of Giessen & Monokaryotic and elongated initials type \\
\hline
\end{longtable}
\arrayrulecolor{black}




\section{Methods}
\subsection{Pre-screening of the secondary metabolite production in the Cyclocybe aegerita mushroom (fungi)}
To prioritize the strains for scale-up, a pre-screening of five Cyclocybe strains was conducted prior to the start of this project by Karen Harms. These five strains were cultured in ZM 1/2 liquid media. Crude extracts obtained from both the supernatant and mycelia were analyzed, yielding consistent results across all strains. Consequently, strain AAE-3, the wild-type strain of Cyclocybe aegerita, was selected for further use.


\subsection{Screening of c.agerita in different culture media}
A preliminary screening was conducted to assess C. agerita in order to chemically describe the production of secondary metabolites. In this study, the fungus was cultivated in nine different media, comprising seven liquid media (ZM 1/2, Q6 1/2, MGP, SYM, MEA, MR, PDB) and two solid media (Rice and OAT).

\subsection{Preparation of pre-culture}
Cyclocybe agerita had sufficiently grown on malt extract agar (MEA) plates, small pieces (9.1 mm in diameter) were excised using a cork borer. Eight of these pieces were transferred into 250 mL sterilized Erlenmeyer flasks containing 100 mL of YM 6.3 medium. The flasks were then incubated at 23°C with shaking at 140 rpm for 7 days. After the incubation period, the cultures were homogenized using a homogenizer set at 8,000 rpm for 10 seconds, repeated twice. These homogenized liquid cultures were subsequently used as inoculum for the main cultures.

\subsection{Preparation of main cultures}
500 ml shake flasks were utilized for the liquid cultivation, each containing 200ml of the respective medium. Hence, 2ml of preculture was added to each flask. The liquid main cultures were incubated in darkness at 23°C with shaking at 140 rpm.\\

To prepare the solid media, 28 g of rice or 35 g of oat were weighed out and put into 500 mL Erlenmeyer flasks. The liquid BRFT medium was then prepared and 100 mL were added to the rice and oat. After autoclaving and cooling, 4 mL of the pre-culture were transferred into each flask. Finally, the cultures were incubated for 57 days at 23°C.


\subsection{Extraction of the secondary metabolites}

\subsubsection{Extraction of secondary metabolites from liquid media} \label{ extraction-liquid-media}
After inoculation, glucose levels in the liquid cultures were regularly checked using Medi-Test glucose strips \cite{Rita2023}. The cultures were harvested and extracted five days after the glucose was depleted. A funnel containing a piece of gaze or a filter paper  was employed to separate the supernatant from the mycelium, and both components were extracted independently.\\


The supernatant was subjected to three rounds of liquid-liquid extraction using ethyl acetate in a 1:1 ratio by separation funnel \cite{Harms2021}. The organic phase was then separated, combined, dried by filtering through anhydrous sodium sulfate, and evaporated under vacuum rotary evaporator, while the water phase was discarded \cite{Harms2021}. The dried organic phase extracts  transferred to 4 mL vials.\\

To maximize the total extract yield, the mycelium was soaked in acetone and treated with 30-minute cycles of ultrasonic bathing at 40°C and an intensity of 100\% for twice \cite{Harms2021}. This process helped release the compounds from the mycelium. After the ultrasonic cycle, the suspensions were filtered to remove any solids, and the liquid was collected in a round-bottom flask. Afterward, the acetone was evaporated, and the obtained extract was extracted against three times ethyl acetate as the supernatant following the protocol above explained. 

\subsubsection{Extraction of secondary metabolites from solid media} \label{extraction-solid-media}
After 57 days of cultivation at a temperature of 23°C , the solid media employed in the study were extracted. The mycelium on the solid surface was covered with acetone and treated with  ultrasonic bathing for 30 minutes at 40°C with 100\% intensity \cite{Harms2021}. Then, the acetone was filtered with a funnel containing a filter to separate from the solid residue, and it was evaporated. The remaining phase was extracted against three parts of ethyl acetate in a separation funnel. The ethyl acetate phase was filtered through a funnel containing filter paper and sodium sulfate to remove any water content. Subsequently, the ethyl acetate was evaporated to dryness.\\

For the rice medium, an additional step was performed. The ethyl acetate crude extract was dissolved in methanol. This dissolved crude extract was then subjected to liquid-liquid extraction in a separatory funnel using a methanol-water mixture (1:1) and heptane \cite{Harms2021}. This step was used to separate fatty acids and other non-polar compounds from the more polar components. The process was repeated with the aqueous phase obtained from the first extraction. Both the aqueous and heptane phases were collected and evaporated to dryness. The resulting crude extract was transferred into 4ml glass vials using methanol mixture and then dried using a nitrogen dryer.





\subsection{ Tissues depending terpenoid production in Monokaryotic and Dikaryotic C.agerita strains}

\subsubsection{Cultivation and extraction of C. agerita Strains to determine tissue-specific production of secondary metabolites}
Small pieces (9.1 mm in diameter) were excised from each fungal strain using a cork borer. One piece of each fungal strain was inoculated into modified crystallizing dishes (Lowe dish: 70 mm in diameter for the lower dish, 80 mm in diameter for the upper dish) containing 16 mL of 1.5\% malt extract agar (MEA). The dishes were sealed with parafilm and placed in a dark incubator at 24°C for 10 days \cite{Orban2020}.\\

After the 10-day incubation period, the parafilm was removed, and the dishes were transferred from the incubator to a transparent box at room temperature for an additional 10 days. After a total of 21 days, the first batch was taken out from the box and extracted. The second batch was extracted after 28 days.

After 21 and 28 days, Batches 1 and 2 were harvested for extraction. The caps and stripes were carefully excised and separated from the agar plates. Each tissue type (caps and stripes) was placed into individual glass vials for extraction. The samples were subjected to two rounds of sonication in an ultrasonic bath with acetone for 30 minutes. Following sonication, the samples were shaken with ethyl acetate, and the ethyl acetate phase was separated. The organic phase was then dried using a nitrogen dryer to concentrate the extract.\\

The overgrown MEA agar plates were segmented using a spatula. These segments were then transferred into a Schott flask and covered with acetone. The flask was kept for 20-minutes in an ultrasonic bath at 40°C with an intensity of 100\%. Subsequently, the acetone was filtered through a filter, and the resulting acetone extract was evaporated \cite{Harms2021}.\\

Next, the acetone phase was extracted against two portions of ethyl acetate by a separation funnel. This ethyl acetate solution was then evaporated until dryness, resulting in an extract transferred to 4 ml glass vials with a mixture of acetone . The transferred extract was further dried using a nitrogen dryer for subsequent procedures.


\subsection{Work procedure in analytical HPLC and Mass spectroscopy (LC-MS)} \label{lc-ms}
The dried crude extracts were dissolved in acetone  to prepare a concentration of  5 mg/ml. From this solution, 150 $\mu l$ was transferred into an HPLC-MS vial for analysis.\\

 An analytical HPLC system (Dionex Ultimate 3000 Series) coupled with an ion trap mass spectrometer (amazon speed™) was employed. The mobile phases consisted of water with 0.1\% formic acid (FA) (A) and acetonitrile with 0.1\% FA (B). A sample volume of 2 $\mu l$ was injected and separated using an ACQUITY-UPLC® BEH C18 column (50 × 2.1 mm, 1.7 $\mu m$ pore size). The gradient program started with 5\% acetonitrile (B) with 0.1\% FA, increasing to 100\% acetonitrile with 0.1\% FA over 20 minutes, followed by a 5-minute hold at 100\%. The flow rate was maintained at 600 $\mu l$/min \cite{Harms2021}.\\
 
The acquired data were processed using Bruker DataAnalysis software (Version 6.1). Compounds were compared with known biological sources using the Dictionary of Natural Products (DNP, Version 30:1, June 2023), allowing for the identification of potential novel compounds based on their biological origin and mass profile.\\


\subsection{Scale up}
\subsubsection{Large-scale cultivation and Isolation of pure compounds from C. agerita  in ZM, MEA and Rice medium}

The medium for scaling up was selected based on the analytical HPLC chromatograms and mass spectra of the crude extracts from the prescreening stage for each media with AAE-3 strain. Among the tested media, the fungal extracts from C. agerita AAE-3 cultured in ZM 1/2 medium, Malt Extract Agar (MEA), and rice medium showed higher yields of  new secondary metabolites than the other media. Therefore, large-scale cultivation in these media was performed to isolate and identify the secondary metabolites.\\

For the scale-up, In ZM medium, a total of 8 L was used, with each of the 1 L flask containing 400 mL of medium and 4 mL of preculture. In MEA medium, 2 L was used and  each 500 mL flask containing 200 mL of medium and 2 mL of preculture was added. In rice medium, 1.2 L were used, with each 500 mL flask containing 100 mL of medium and 5 mL of preculture. The ZM ½, MEA, Rice cultures were extracted after 61 days, 31 days, 57 days. The cultivation and extraction procedures were identical to those described in section  \ref{ extraction-liquid-media} and \ref{extraction-solid-media}.


\subsection{Extraction of the big scale fermentation}
Separation of the mycelium from the supernatant was done using a funnel with a gaze. The mycelium and supernatant were extracted separately using different methods, followed by their separation. Mycelium which was grown in liquid media and BRFT rice media cultures extraction procedures were identical to those described in section \ref{ extraction-liquid-media} and \ref{extraction-solid-media}.\\

For supernatant, the crude supernatant extracts from MEA cultures were obtained in 2 L, following the extraction procedure described in section \ref{ extraction-liquid-media}. However, From the ZM ½  cultures, volume of 6.5 L of supernatant was obtained. Due to the large volume extraction by shaking method was not possible. For this reason, an alternative method was employed to extract the supernatant from ZM ½ cultures.\\

\subsubsection{ZM ½  cultures supernatant extraction using polymeric resin adsorbent}
In this method, polymeric resin Amberlite® XAD16N (Sigma-Aldrich, Darmstadt, Germany) was used, previously washed with distilled water (1\% w/v), methanol (1\% w/v), and distilled water (0.5\% w/v) again for 30 minutes with each solution, respectively \cite{CharriaGirn2021}. A concentration of 6g resin/l was applied and mixed with supernatant in an 10L schott bottle and shaked using agitation plates (100 rpm) at room temperature for 24 hours \cite{Caicedo2010}. After 24 hours, the resin was filtered using a sieve and organic solution was analyzed by Lc-Ms. No detectable organic compounds were contained, indicating complete bond of the compounds onto XAD resins. Then dried XAD resins packed into an open column and washed with a series of solvent gradients as shown in Table \ref{table1A}. Four fractions were collected, with fraction number three was selected for further analysis.\\

Afterwards, crude extracts from supernatant and mycelia were transferred to 4-mL vials using appropriate solvents (acetone: MeOH, Ethyl acetate, aceton, methanol, deionized water) and were dried in a nitrogen flow at 40 °C. Eventually, the crude extracts were analyzed in an AmaZon speed ESI-ITMS with dual ion funnel technology as described in section \ref{lc-ms}. The MS chromatograms were evaluated with Bruker Analysis Software (DataAnalysis 6.1). The crude extracts with an optimum yield and promising MS profiles were selected for further separations.

\subsection{Stability test of the crude extracts}
To test the stability of the compounds in the crude-extracts across different solvents, a small portion of crude extract (4.5mg) was transferred into 9 vials and dried them under a nitrogen dryer. Each vial was then covered with 1ml of the solvent to be tested. Eight vials were placed in a heat block at 40 °C for 24 hours while one vial measured immediately as a control \cite{Harms2021} . After 24 hours, the samples were analyzed by LC-MS as described in section \ref{lc-ms}. The resulting data were compared to both control sample and tested sample using Data Analysis 6.1. Compounds were considered unstable in the tested solvent if significant differences in peak areas were observed after the test. If the test solvent showed overlapping peaks with the control, it was considered suitable for further separations \cite{Harms2021}.



\subsection{Isolation of pure compounds}
\subsubsection{Normal phase liquid chromatography}
First the crude extracts were initially analyzed by TLC plate to assess which solvents showed optimal separation. The crude extract from MEA supernatant and mycelium were initially subjected to normal phase Flash chromatography. The fraction was achieved using the Reveleris®X2 normal phase HPLC system equipped with a polar silica cartridge (Buchi Flash Pure) with weight 5g \cite{Rita2023}. The mobile phase  consisted of three solvent mixtures supplemented with 0.1\% formic acid: hepten(100\%), Chloroform-TBME(3:1) and Acetone-Methanol(1:1). These solvent systems were introduced with increasing polarity and a three-step gradient elution, as described in Table \ref{table2A}.\\

Similarly, crude extracts from rice media were subjected to normal-phase flash chromatography. The fractionation was performed using a Reveleris® X2 HPLC system with a 12 g polar silica cartridge (Büchi FlashPure) \cite{Ellens}. Additionally, vacuum liquid chromatography was performed to further separate the total extract from solid rice culture of C. agerita (330 mg). The total extract was dissolved in a minimal amount of heptane-dichloromethane-methanol (1:1:1) and allowed to dry on 15 g of silica gel as sand . The dried silica gel was then carefully packed on top of a column filled with dry silica gel for vacuum liquid chromatography. The mobile phase for this method was made up of three solvent mixtures, each containing 0.1\% formic acid: heptane (100\%), ethyl acetate-heptane (2:1), and acetonitrile-methanol (1:1). Non polar compounds eluted early in this separation process.\\

ELSD and UV/Vis detectors were used to detect and collect fractions, allowing the identification of non-chromophoric and low-yield chemicals in crude extracts \cite{Ellens}. The gradient elution, as described in Table \ref{table3A}, resulted in the collection of seven 12ml fractions, which were each evaporated to dryness under reduced pressure.



\subsection{Reverse phase liquid chromatography}
To purify the secondary metabolites of C.agerita AAE-3 strain, the fractions obtained from flash chromatography or open column chromatography were subjected to preparative HPLC. Prior to this, analytical reverse-phase HPLC (Agilent Technologies 1260 series) was performed using different columns according to the properties of the compounds to determine the optimal gradient for preparative HPLC separation. Ultra-pure water (A) and ACN (B) with 0.1\% formic acid were used as mobile phase. The information from the LC-MS chromatograms and the analytical HPLC guided the development of a purification gradient, with gradient steps extended by 5 to 20 minutes for Preparative HPLC. The Gilson System (Middleton, Wisconsin, USA) was equipped with GX-271 Liquid handler, a DAD 172 together with a 305 and 306 pump was carried to perform the preparative separations \cite{Harms2021}. The flow rate and the column were selected based on analytical HPLC results. Each crude extract was separated with a different gradient (see in appendix Table \ref{table1A} to Table \ref{table9A}). The fractions were collected in tubes by volume and combined based on UV profiles.\\

After the separation, acetonitrile and water were evaporated to dry ACN phase using rotary evaporator. The watery phase was separated with ethyl acetate and then the ethyl acetate phase was dried. Finally, the dried ethyl acetate phase was dissolved with acetone or methanol and  transferred to 4 mL glass vials and dried in the nitrogen dryer for storage. The purity of the fractions was analyzed with LC-MS, as described in section \ref{lc-ms}.\\

Furthermore, the fractions which were not purified with low yield were separated on a semi-preparative HPLC system (Agilent 1200 series) using similar mobile phase. First 2 $\mu L$ of the fraction was subjected to the column to determine the elution time of the compounds. After getting the good separation with the right column and optimal gradients, fractionation was accomplished by a time dependent collection using a G1364C fraction collector. After collected fractions  were processed using the same protocol employed following fraction collection from the Gilson system.\\

Accordingly, seven different compounds (1-7) were isolated as described in the result section. The HPLC conditions are indicated in Appendix ( Table \ref{table4A}-\ref{table9A})



\subsection{Isolation of pure compounds 1-3 from C.agerita in ZM ½ culture}
The separation of crude extracts from ZM media using the open column chromatography obtained in four fractions. Fraction three (AAE-3-ZM-F3) amount was 745 mg and was selected  for isolation by preparative reversed phase HPLC . Due to its large quantity, it was divided into three parts and subjected to preparative reversed-phase HPLC separations three times under the conditions listed in Table \ref{table4A}.\\

The first portion used for separation obtained 21 fractions, the second portion produced  25 fractions and the third portion generated 22 fractions. Compound 1 was obtained from run 3 fraction 20 ( 12.44 mg). The fractions from three HPLC runs were analyzed and compared. Fractions  containing compounds with identical masses were combined and the total mass was measured. Further purification was performed using the Gilson system.\\

Compounds 2 isolated from run 1( Fraction 12), run2 (fraction 17) and run 3( fraction 13). The fraction from all three runs  were combined and obtained a total of 17 mg. This combined amount was utilized to further purification using a flow rate 20 ml/min and Waters X-Bridge C18 column (with a size of  250*10mm, 5µm). After purification, compound  2 was obtained with a final mass of  3.62 mg.The HPLC  gradient employed for this process  listed in Table \ref{table5A}.\\

Compound 3 was isolated from ZM media but not during the upscaling process. It was obtained from a small scale extraction using the same preparative reverse-phase HPLC and gradient conditions, with the modification of using solvents with 0.1\% formic acid. The method condition was described in Table \ref{table6A}, with a flow rate of 40 ml/min and a Gemini® 10 µm C18 column (with a size of 250 × 21 mm). The  injection mass was 250 mg which dissolved in ethyl acetate. The purification resulted in the isolation of one pure compound, identified as compound 3, found in fraction 19.


\subsection{Isolation of pure compounds 4 and 5  from C.agerita in MEA cultures}
The separation of crude extracts from MEA media using Buchi Flash Pure chromatography resulted in  seven fractions. The first three fractions were found to be similar and combined, yeilding 70 mg( AAE-3-MEA) from these three fractions. This 70mg amount was subjected to further purification using preparative reverse-phase HPLC separation conducted in two times. The method outlined in Table \ref{table7A}, applying a flow rate of 20 ml/min with a Nucleodur C18ec column (125*2mm, 5µm ) and an injection mass of 35mg. As a result, two pure compounds were isolated such as compound 4 (17 mg)  and compound 5 (10 mg).


\subsection{Isolation of pure compounds 6, 7 and 8 from C.agerita in Rice cultures}
The separation of crude extracts from rice media using Buchi Flash Pure chromatography resulted in eight fractions. Mass spectrometry (MS) analysis of these fractions led to the selection of Fractions 6 and 4 for further purification, due to their high yield and the presence of high-intensity peaks. Fraction 6 was subjected to preparative reverse-phase HPLC, resulting in the isolation of two compounds: compound 6 (2 mg) and compound 7 (0.94 mg). The HPLC conditions for this purification was listed in Table \ref{table8A}, using a flow rate of 50 mL/min with a Luna C18 column (250 × 50 mm, 10 µm) and an injection mass of 118 mg which dissolved in methanol.\\

Fraction 4 (15mg) was subjected to semi preparative reverse-phase HPLC for further purification. Due to the large quantity, the 15 mg sample was divided into three parts, each portion using the same gradient to conduct three separate runs. The fraction from these runs were collected and combined, resulting in the isolation of compound 8. The HPLC method was listed in  Table \ref{table9A}, employing a flow rate 5 ml/min with a Luna C18 column (250*10mm, 5µm) and an injection mass of 5 mg in each run which dissolved in ethyl acetate.


\subsection{Structure elucidation}
The Bruker Avance III 500 (1H: 500 MHz, 13C: 125 MHz) and Bruker Avance III 700 with a 5 mm TXI cryoprobe (1H: 700 MHz, 13C: 175 MHz) spectrometers were utilized for the 1D and 2D NMR analyses \cite{Ellens}. These analyses were executed at the NMR facility of Helmholtz Centre for Infection Research by Esther Surges and Christel Kakoschke. The samples were dissolved in DMSO-d6 or acetone-d6 to facilitate the analysis. An ACQUITY UPLC® BEH C18 column (130Å, 1.7 µm, 2.1 mm × 50 mm; Waters) was used in conjunction with an Agilent 1200 series HPLC-UV system to evaluate pure compounds (at a concentration of 1 mg/mL in acetone:methanol) \cite{Ellens}. This system was coupled with an Electrospray Ionization Time-of-Flight Mass Spectrometry (ESI-TOF-MS), operating within a scan range of 100–2500 m/z, a rate of 2 Hz, a capillary voltage of 4500 V, and a dry temperature of 200 °C using the Maxis instrument from Bruker to yield high-resolution electrospray ionization mass spectra \cite{Ellens}. Structure elucidation of the isolated compounds was concluded by Dr. Frank Surup and Karen Harms, a fellow postdoc and PhD student in the department of Microbial Drugs group based on exhaustive spectral analyses including HRESIMS, 1D (1H/13C) and 2D (1H-1H COSY, HMBC, HSQC and ROESY) NMR spectroscopy.


\subsection{Tendem mass spectrometry (MS/MS) measurements and data analysis}
Total extracts from the extraction were used to prepare samples for tandem mass spectrometry. For this, a Dionex Ultimate 3000 RS system with a Kinetex C18 column (1.7 $\mu$m, 21 × 150 mm, 100 Å) 
and a 2 $\mu$L injection volume was used to analyze each sample, dissolved at a concentration of 450 $\mu$g/mL in DMSO. The flow rate of the mobile phase consisting of A (\ce{H2O} + 0.1\% formic acid) and B (acetonitrile + 0.1\% formic acid) was 0.3 mL/min. Over the course of 25 minutes, the gradient changed from 1\% B to 100\% B. UV data (190–600 nm) was collected, and MS spectra were acquired using a timsTOF Pro 2 mass spectrometer (Bruker Daltonics) with parameters: spectra rate 9.52 Hz, PASEF on, cycle time 320 ms, MS/MS scans 2, scan range (m/z 100–1800 Da). Data was acquired in positive and negative ESI modes \cite{Rita2023}. Using MetaboScape® 2022, the data was pre-processed to exclude any blank features and the remaining features based on accurate molecular weight with NPatlas databases. Compound hits were validated using MetaboScape, considering predicted molecular formulae and fragmentation patterns \cite{Wongkanoun2023}. GNPS tools used to directly match their data to all public MS/MS reference libraries for annotation \cite{Wang2016}. SIRIUS allows for the automated and high-throughput analysis of small-compound MS data beyond elemental composition without requiring compound structures or a mass spectral database \cite{sirius}.




\subsection{Spectral analysis of pure compounds}
\subsubsection{UV/Vis absorption}
A UV-2450 UV/Vis spectrophotometer (Shimadzu Corporation) was employed to measure the compounds' UV absorbance properties between 190 and 600 nm. The compounds were dissolved in methanol Uvasol® at a final concentration of 0.01 mg/mL. Two cuvettes were used in the experiment: one held 1 mL of methanol Uvasol® as the reference, and the other 1 mL of the respective test compound solution. Prior to measurement, baseline correction was performed, followed by auto-zeroing \cite{Ellens}.  To identify the compounds' absorbance maximum values within the designated wavelength range, UV absorption experiments were conducted. The extinction coefficient ($\epsilon$) was calculated as following: \[\epsilon = \frac{A}{c * l}\]
$\epsilon$: molar extinction coefficient (in L/mol*cm)\\
A: absorbance of the sample\\
c: concentration of the absorbing species (in mol/L)\\
l: path length of the cuvette (in cm)

\subsubsection{Circular dichroism}
Circular dichroism was measured for the compounds (in MeOH Uvasol®) using a J-815 CD spectrometer (Jasco) \cite{Rita2023}. First, the baseline was measured using the MeOH Uvasol ® in the wavelength range of 190 to 600 with 5 repetitions \cite{Ellens}. The compounds were dissolved in methanol Uvasol® at a final concentration of 0.25 mg/mL. Furthermore, the measurement was carried out for compounds in the same wavelength range and number of repetitions.

\subsubsection{Optical rotation}
The optical rotation of the compounds was measured on an Anton Paar MCP 150 polarimeter (sodium D line, nickel alloy sample cell 100 mm × 3 mm, 0.7 mL) \cite{Rita2023}. First, the device was calibrated using the quartz plate. After the successful calibration, auto-zeroing was carried out with MeOH Uvasol®, the solvent used for dissolving the compounds \cite{Ellens}. The optical rotation was measured at the concentration of 1 mg/mL. The specific optical rotation values were calculated using the following formula:\[[\alpha]^{T}_{\textsc{D}} = \frac{\alpha * 100}{c * l}\] 

[$\alpha$]: specific rotation\\
$\alpha$: measured optical rotation\\
T: temperature (20 °C) l: 1 dm (length of the sample cell)\\
D: sodium D line (589 nm) c: concentration of the sample (g/100mL)




\subsection{Bioactivity testing}
The bioassays were performed by Wera Collisi from the Microbial Drugs Department at the Helmholtz Centre for Infection Research. To assess the antimicrobial activity of pure compounds against bacterial and fungal strains, the minimum inhibitory concentrations (MIC) were determined using a serial dilution method (Table \ref{table-bacteria}). The test compounds were dissolved in methanol (MeOH) to a final concentration of 1 mg/mL.\\

Bacterial and fungal cultures were grown overnight in 50 mL shaking flasks containing 25 mL of medium at 140 rpm. The following day, the optical density (OD) was measured to monitor growth (OD600 nm for most bacteria, OD548 nm for fungi and Mycobacterium smegmatis) \cite{Ellens}. If sufficient growth was achieved, aliquots were stored at -80°C in 1.5 mL reaction tubes.Prior to testing, aliquots were thawed, and their OD was adjusted using specific media to OD600 nm of 0.01 for most bacteria and OD548 nm of 0.1 for fungi and M. smegmatis \cite{Harms2021}. For the assay, 150 µL of the adjusted cultures were added to 96-well round-bottom microtiter plates. In row A, 130 µL of the suspension was added, followed by 20 µL of the test compounds (1 mg/mL) in columns 1 to 10. Positive and negative controls were added to columns 11 and 12, respectively \cite{Harms2021}. Positive controls varied based on the organism tested (Table \ref{table-bacteria}), with MeOH serving as the negative control.\\

A serial dilution was performed by transferring 150 µL from row A to the subsequent rows, continuing until row H, where the remaining 150 µL were discarded. This resulted in final concentrations of the test compounds ranging from 66.6 µg/mL in row A to 0.5 µg/mL in row H . The plates were incubated overnight on a microplate shaker at 800 rpm, at either 30°C or 37°C, depending on the organism tested. MIC values were determined after 24 hours \cite{Harms2021}. Specific test strains and conditions are detailed in Table \ref{table-bacteria}.\\


% \begin{center}
% \begin{tabular}{ |c|c|c|c|c| } 
%  \hline
%  Organism & Strain number & Growth media & Positive control (Reference) & Temperature \\ 
%     \hline
    
%  cell4 & cell5 & cell6 & cell3 & cell3 \\ 
%  cell7 & cell8 & cell9 & cell3 & cell3 \\ 
%  \hline
% \end{tabular}
% \end{center}

\arrayrulecolor{black}
\begin{longtable}{!{\color{black}}>{\hspace{0pt}}m{0.265\linewidth}!{\color{black}}>{\hspace{0pt}}m{0.154\linewidth}!{\color{black}}>{\hspace{0pt}}m{0.152\linewidth}!{\color{black}}>{\hspace{0pt}}m{0.173\linewidth}!{\color{black}}>{\hspace{0pt}}m{0.149\linewidth}!{\color{black}}}
\caption{The bacterial and fungal strains and test conditions applied in the antimicrobial testing} \label{table-bacteria}\\ 
\hline
\textbf{Organism}                        & \textbf{Strain\newline number} $^1$  & \textbf{Growth\newline media} & \textbf{Positive\newline control\newline(Reference)} & \textbf{Temperature}  \endfirsthead 
\hline
Schizosaccharomyces pombe      & DSM 70572     & MYC          & Nystatin\par{}1.0 mg / mL        & 30            \\ 
\hline
Pichia anomala                  & DSM 6766      & MYC          & Nystatin\par{}1.0 mg / mL        & 30            \\ 
\hline
Mucor hiemalis                  & DSM 2656      & MYC          & Nystatin\par{}1.0 mg / mL        & 30            \\ 
\hline
Candida albicans                & DSM 1665      & MYC          & Nystatin\par{}1.0 mg / mL        & 30            \\ 
\hline
Rhodotorula-glutinis            & DSM 10134     & MYC          & Nystatin\par{}1.0 mg / mL        & 30            \\ 
\hline
Micrococcus luteus              & DSM 1790      & MHB          & Oxytetracycline\par{}0.1 mg / mL & 30            \\ 
\hline
Escherichia coli                & DSM 1116      & MHB          & Oxytetracycline\par{}0.1 mg / mL & 37            \\ 
\hline
Bacillus subtilis               & DSM 10        & MHB          & Oxytetracycline\par{}1.0 mg / mL & 30            \\ 
\hline
Mycobacterium\par{}smegmatis    & ATCC-700084   & 7H9+ADC      & Kanamycin\par{}0.1 mg / mL       & 37            \\ 
\hline
Staphylococcus aureus           & DSM 346       & MHB          & Oxytetracycline\par{}0.1 mg / mL & 30            \\ 
\hline
Pseudomonas aeruginosa          & DSM 19882     & MHB          & Gentamicin\par{}0.1 mg / mL      & 37            \\ 
\hline
Chromobacterium\par{}violaceum & DSM 30191     & MHB          & Oxytetracycline\par{}0.1 mg / mL & 30            \\
\hline

\end{longtable}
\arrayrulecolor{black}
{\footnotesize{$^1$ DSM. Leibniz-Institut DSMZ – German Collection of Microorganisms and Cell Cultures GmbH, Braunschweig, Germany; ATCC. American Type Culture Collection 2MHB. Müller-Hinton Broth; MYC. 1\% bacto peptone, 1\% yeast extract, 2\% glycerol, pH 6.3; 7H9+ADC. Middlebrook 7H9 Broth Base + Middlebrook ADC Supplement}}\\



The cytotoxicity assay was conducted using a 96-well plate. The L929 and KB3.1 cell lines were incubated at 37°C with 10\% CO2 in Gibco™ Dulbecco’s Modified Eagle Medium (DMEM), supplemented with 10\% Fetal Bovine Serum (FBS). Each well of one plate was filled with 120 µL of a cell suspension (50,000 cells/mL), while a second 96-well plate was prepared with 100 µL of growth media in each well \cite{Harms2021}.\\

Next, 50 µL of the test compounds (1 mg/mL) were added to the first column of the second plate, with each compound tested in duplicate. Methanol (MeOH) and a MeOH (9:1) mixture served as negative controls \cite{Harms2021}. A serial dilution was performed by transferring 50 µL from the first row to the subsequent rows, continuing until the last row, where the remaining 50 µL were discarded. 
This resulted in concentrations ranging from 333 µg/mL to 1.9 x $10^{-3}$ µg/mL.\\

From this dilution plate, 60 µL of the solution (ranging from 111 µg/mL to 1.9 x $10^{-3}$ µg/mL) were added to the first plate containing the 120 µL of cell suspension, achieving final concentrations from 37 µg/mL to 0.6 x $10^{-3}$ µg/mL. After 5 days of incubation, the half-maximal inhibitory concentration (IC50) was determined using a colorimetric MTT assay. For this, 20 µL of a 5 µg/mL MTT solution (3-(4,5-dimethyl-2-thiazolyl)-2,5-diphenyl-2H-tetrazolium bromide) was added to each well and and incubated for 2 hours at 37°C. The plates were then centrifuged at 3,000 rpm for 5 minutes, the supernatant removed, and the wells washed with 100 µL of phosphate-buffered saline (PBS)  \cite{Harms2021}. The cell cultures were carried out in the respective appropriate media (Table 8)\\

The well plate was centrifuged again, and the supernatant was removed. Then, 100 µL of (Isopropanol:HCl) solution was added to each well. After incubating for 10 minutes at room temperature, the absorbance at 595 nm was measured using a microplate reader. The IC50 value was determined from the absorbance plot and converted to µM units \cite{Harms2021}. If a cytotoxic effect was observed in these cell lines, additional cell lines were tested using the same protocol \cite{Becker2020}.



% \usepackage{color}
% \usepackage{tabularray}
\definecolor{Black}{rgb}{0,0,0}
\begin{longtblr}[
  caption = { Cytotoxicity assay Cell lines and experiment parameters},
]{
  width = \linewidth,
  colspec = {Q[440]Q[102]Q[117]Q[275]},
  hlines,
  vlines = {white},
  vline{1} = {-}{white},
}
\textbf{Type} & \textbf{Cell line} & \textbf{Number$^1$} & \textbf{Growth media$^2$}\\
Mouse fibroblasts & L929 & ACC2 & DMEM + 10 \% FBS\\
Human endocervical Adenocarcinoma (AC) & KB 3.1 & ACC 158 & DMEM + 10 \% FBS\\
Human prostate AC & PC-3 & ACC 465 & F-12K Nutmix + 10 \% FBS\\
Human ovary AC & SK-OV-3 & n / a & {McCoys 5a + 10 \% FBS\\RPMI 1640 + 10 \%}\\
Human breast AC & MCF-7 & ACC 115 & {+ 1 \% MEMNEAA\\+ 1.25 mL / 500 mL insulin}\\
Human squamous AC & A431 & ACC91 & RPMI 1640 + 10 \% FBS\\
Human lung carcinoma & A549 & ACC107 & DMEM
\end{longtblr}
{\footnotesize{$^1$ ACC. Leibniz-Institut DSMZ – German Collection of Microorganisms and Cell Cultures GmbH, Braunschweig, Germany\\
$^2$ DNEM. Dullbecco’s Modified Eagle Medium; FBS. Fetal Bovine Serum; F-12K Nutmix. 
Ham’s F-12K (Kaighn's) Medium; McCoys 5a. McCoy’s 5a (modified) Medium; RPMI 1640. 
RPMI 1640 Medium; MEM NEAA. MEM non-Essential Amino Acids Solution 100x; Insulin.
Human Recombinant Insulin, Zinc Solution}}\\
